\documentclass[12pt,a4paper]{article}
\usepackage{geometry}
\geometry{left=2.5cm,right=2.5cm,top=2.5cm,bottom=2.5cm}

\usepackage[utf8]{inputenc}
\usepackage{listings}
\usepackage{CJK}
\usepackage{xcolor}
\usepackage{graphicx}


\begin{document}
\begin{CJK}{UTF8}{gbsn}
\title{走 出 误 区   真 正 黑 客 是 门 艺 术 }
\author{ismdeep}
\date{2013-10-09 23:28:34}


\maketitle

%设置listings
\lstset{numbers=left,
numberstyle=\tiny,
keywordstyle=\color{blue!100}, commentstyle=\color{red!50!green!50!blue!50},
frame=single,tabsize=4,showtabs=false,extendedchars=false,
rulesepcolor=\color{red!20!green!20!blue!20}
}

\begin{abstract}
黑客这个词从诞生到现在,从来就没有解释为“高级入侵者”、“病毒制造者”或者“QQ盗号者”过。我至今不清楚在中国是谁先把黑客和这些无聊的词汇联系在了一起,导致如此多的人被误导。但有一点是肯定的,某些不负责任的媒体一直侮辱这两个字。
\end{abstract}


黑客这个词从诞生到现在,从来就没有解释为“高级入侵者”、“病毒制造者”或者“QQ盗号者”过。我至今不清楚在中国是谁先把黑客和这些无聊的词汇联系在了一起,导致如此多的人被误导。但有一点是肯定的,某些不负责任的媒体一直侮辱这两个字。

那个写熊猫烧什么来着的叫什么来着的也叫黑客?太可笑了。他做鬼也当不了黑客。请分清这两个词,hacker \& cracker,别去研究他们的中文翻译。

不花力气多解释,不懂黑客是什么意思的人终究不懂,我的这篇文章也不是呼吁那些垃圾更正他们的思想(请不懂的人关闭网页窗口),而是写给那些真正怀揣着黑客梦想的人看的。

首先,有这样的梦想的人出于两种目的(不可能有第三种),真心喜欢计算机技术的,和为了炫耀的。(后者请关闭这个网页窗口)

就像我前面所说的,黑客不一定要写病毒,不一定要入侵,更不会无聊到盗号。可能你会笑了,入侵是黑客的天性,其实这是错误的。请各位去查看黑客这个词在20世纪60以及70年代的解释。“热爱探索问题,解决问题的一批人”,“热爱编出精妙程序的人”。黑客根本不是一种技术境界,而是思想境界,它是一种文化,是一种精神(我把它理解为生活方式,就像hip-hop一样),是一门艺术。

我是因为喜欢黑客技术才接触Linux的,却发现更多的国人热衷于在盗版的Windows下用着别人写出来的软件扫什么端口,查什么漏洞,入侵什么网,会按以上这些按钮后,大肆炫耀一番。这些人普遍智力水平低下(对不起,黑客技术到最后就是拼智力),只会用右手[注1]。Windows从来就是被攻击的目标系统,而不是攻击者应该使用的系统,更不会是黑客的玩具。那么什么才是呢?

\begin{LARGE}
Unix
\end{LARGE}

不知道中国那么多会写病毒,会入侵的人了解、知道甚至听说过这个东西。Unix一整套的设计理念以及哲学还有发展史就代表着黑客这两个词。

Unix诞生于1969年,1969年的东西,流传至今的有哪些?很遗憾,无论是软件还是硬件,除了Unix与创造它的C语言[注2],没有了。为什么Unix生命力会如此旺盛?答案只有一个,在于它的黑客文化与哲学。

哲学是门高深的学问,我们不需要去很彻底的研究它,Unix的哲学就是4个英文字母:K.I.S.S[注3],这也是最核心的设计理念,Unix有许许多多优秀的哲学思想,其中这个是最重要的。用最简单的东西去完成最复杂的东西,这也就是为什么许多没有玩过Unix从而没有机会接触Perl语言的人无法体会这个道理的原因[注4]。Unix的设计者们全是懒人,正是因为这种懒,系统变的简单易用,稳定无比[注5],正是因为这种懒,缔造了一个不朽的传奇。

那么Unix是如何维持如此旺盛的生命力的呢?答案就是——open source movement,开源运动,Unix最初把源码分发给了各大高等学府用于研究,这些学府各自作了修改,发展出了许多不同类型的Unix,但其本质都是差不多的。随着Richard Stallman[注6]建立了FSF并且推出了GNU Project[注7],开源社区兴起,互联网的飞速发展[注8],越来越多的来自全世界的黑客成为了Unix文化和技术的继承者。

Linus Torvalds就是其中一位

Linus做梦也不会想到,当初只是开放一个自己编写的Unix-like的系统源代码会让他有资格在世界的舞台上和Bill Gates同台竞技[注9]。Linux的诞生绝对不是偶然,而是一种必然。

有点偏题了,现在我想问一个问题就是,如果你使用的是Windows,你了解你的系统吗?敢说了解的只有两种人,有权力看代码的微软工作人员和狂妄的人。你有没有为你不知道某个文件或者文件夹到底是干什么的而头疼不已?有没有为无法自己修改系统的某些臃肿的功能而烦恼?有没有为管理员的权限也无法删除某些不必要的系统文件而恼火?如果有,你具有黑客精神,但你用玩错了玩具,如果没有,请关闭这个窗口。

Windows是Business-Desktop Product,是用来卖钱的,可惜大多数国人素质低下,没有去维护他人权益的思想和意识[注10],买来的都是盗版,盗版又怎么样呢?Windows终究是Windows,它不会因为盗版而让你把它的技术公布出来,微软放任着中国盗版很大的一个原因是,他知道中国人的德性,先让你用习惯了,当你习惯了以后,它开始打击盗版[注11],你除了Windows什么都不会,甚至脑子里根本没有还有别的操作系统的概念,就必须花钱买正版,这时你就完了,因为你吸上毒了。

所以,更不用谈你能够从这个系统中学到些什么真正的黑客技术尤其是精神还有文化,右手谁都有。

\begin{LARGE}
黑客的思想
\end{LARGE}

接下来讲下黑客文化的一些核心思想。

探索:遇到一个问题后,怎么去解决,几种方法解决,哪种最有效率。如果只是满足于把遇到的问题解决了就OK了的话,那是远远不够的。我们不仅要how-to,还要why-to,这里简单举个例子,虽然我已经快一年没用Windows了,但无聊的时候突发奇想,写了十个在Windows下关闭窗口的方法:

  1、单击右上角大X
  2、双击标题栏最左边的图标
  3、右键单击标题栏选择关闭
  4、文件-退出
  5、Alt-F4
  6、右键单击任务栏,选择关闭
  7、Ctrl+Alt+Del,结束相应任务
  8、Ctrl+Alt+Del,删除相应进程
  9、按主机电源键3秒
  10、把电脑扔出窗外
你可以一笑了之,这只是一个例子,我们在利用多种方法解决同一个问题时,可能会遇到更多的问题,这样,你可以学到更多东西,如果你懒的解决多余的问题,那么关闭这个窗口。

创造:这是最关键的,偏偏是中国人现在最大的问题,我们在从一年级(甚至从幼儿园)就被教育要循规蹈矩,按常理解决问题,对待事务,我不得不对这种教育制度说,发克油!我有个妹妹才上一年级,问我,月亮像什么,我说帽子,她说错,书本上写的是香蕉,我听了后也不想多说什么,就跟她说,不要做个听话的孩子,我不知道现在的孩子还有多少是说的出大海倒过来就是蓝天之类的话了[注12]。没有创造力,你模仿的再好也没用,社会,包括你的技术不会因为你的模仿能力强而进步。

分享与合作:黑客的技术成长90%要靠自己,10%要靠与他人的交流,分享与合作。不懂得合作、分享的人永远是井底之蛙。最好的例子就是多看别人写的源代码,这是公认的提高技术最快的方法,但是问题在于我们应该怎么去看,怎么去学,还是那句老话,模仿是不会让你的技术有任何进步。


\begin{LARGE}
道德准则
\end{LARGE}


谦虚,友好,热情,还有很多,这些是一个人的品德问题,学黑客先学做人,不多说了,列出比较标准的黑客守则:

  (1) Never damage any system. This will only get you into trouble.

  不恶意破坏任何系统, 这样做只会给你带来麻烦。恶意破坏它人的软件或系统将导致法律刑责, 如果你只是使用电脑,那仅为非法使用!!注意:千万不要破坏别人的软件或资料!!

  (2) Never alter any of the systems files, except for those needed to insure that you are not detected, and those to insure that you have access into that computer in the future.

  绝不修改任何系统文件,除非你认为有绝对把握的文件,或者要改那些文件是为了使你自己在以后更容易的再次进入这个系统而必须更改的。

  (3) Do not share any information about your hacking projects with anyone but those you'd trust.

  不要将你已破解的任何信息与人分享,除非此人绝对可以信赖。

  (4) When posting on BBS's (Bulletin Board Systems) be as vague as possible when describing your current hacking projects. BBS's CAN be monitered by law enforcement.

  当你发送相关信息到BBS(电子公告板)时,对于你当前所做的黑事尽可能说的含糊一些,以避免BBS受到警告。

  (5) Never use anyone's real name or real phone number when posting on a BBS.

  在BBS上Post文章的时候不要使用真名和真实的电话号码。

  (6) Never leave your handle on any systems that you hack in to.

  如果你黑了某个系统,绝对不要留下任何的蛛丝马迹。(绝对不要留下大名或者是绰号之类的,这时由于成功的兴奋所导致的个人过度表现欲望会害死你的。)

  (7) DO NOT hack government computers.

  不要侵入或破坏政府机关的主机。

  (8) Never speak about hacking projects over your home telephone line.

  不在家庭电话中谈论你Hack的任何事情。

  (9) Be paranoid. Keep all of your hacking materials in a safe place.

  将你的黑客资料放在安全的地方。

  (10) To become a real hacker, you have to hack. You can't just sit around reading text files and hanging out on BBS's. This is not what hacking is all about.

想真正成为黑客,你必须真枪实弹去做黑客应该做的事情。你不能仅仅靠坐在家里读些黑客之类的文章或者从BBS中扒点东西,就能成为黑客,这不是“黑客”的真正含义。

说了那么多,还有一个最关键的就是,爱国,不要去黑自己国家的网站,中国人最大的问题就是不团结,谁都想当老大,谁都不服谁,自私是中国人的劣根。朋友,懂得teamwork吧!真正爱国的人不会一天到晚说日本人的不是,美国人的不是,而是珍惜时间,学好技术随时为祖国作贡献的人[注13]。

如果说Windows是属于美国人的,那么Linux就是属于全人类的。你可以在Linux下做你任何想做的事,自由就是唯一的规则。

最后还是要说,黑客是种精神,你不需要拥有顶极的技术,但只要你真正具有这样的精神,你可以自豪的说自己是黑客[注14]。So,看到这里还没有关闭窗口的人,我相信你们会成功。

推荐一些优秀的参考资料:

《Revolution OS》,一部讲述Linux与开源运动的电影,与其说是电影,不如说是纪录片,是由一些采访和阐述组成的,世界顶极黑客Richard Stallman,Eric Raymond等人在片中对黑客的文化,精神,以及历史作了详细解释与说明,强烈推荐。

《The Art of Unix Programming》,这本书为Eric Raymond所作,编写历时五年,汇集了13位Unix先驱的评论,是经典中的经典,讲述的更多的不是技术而是Unix的黑客哲学,我看的是原版,因为网上的评价是翻译的不好[注15]。

\begin{LARGE}
P.S
\end{LARGE}

实在看不下去中国所有的黑客的网站除了一些基本的网络只是几乎全是讲Windows下的工具使用、入侵。说的直接点,不懂Unix的人不懂电脑,更别说懂黑客了,这句话一点也不偏激。

  注1:右手代表着很多意思,他们只会用左脑或者可以理解为他们只会用鼠标,等等。

  注2:在1969年并没有C语言,最初的Unix是由汇编与B语言一起写出来的,后来在Ken Thompson与Dennis Ritchard发明了C后由C重写了Unix的代码。

  注3:Keep It Simple , Stupid。

  注4:Perl的原代码除了在编写好的一个月以内编写者本人看的懂以外,别人根本不可能在没有注释的情况下看懂(编写者本人如果一个月不去维护,一个月后自己看不懂),这种说法毫不夸张,有许多这样的例子。

  注5:别跟我说按鼠标方便,我会举出一大堆例子让你反悔,稳定性更不必说。

  注6:这个人我不想多介绍,我最喜欢的黑客,世界公认的顶极程序设计师。

  注7:Free Software Foundation:自由软件基金会,GNU:GNU's Not Unix。

  注8:网络和Unix是穿同一条开裆裤长大的,黑客们发展了网络,发展了Unix。

  注9:Linus为Linux做的贡献非常小,在我看来,他只是撒了种子这么简单,真正灌水施肥的是来自全世界的无数优秀无比的黑客。

  注10:连维护自己的知识产权的意识都没有,这是相当不好的氛围。

  注11:现在听说什么盗版验证什么的,在右下角会出现的那个东西就是微软采取的措施,它随便打个防伪的补丁就可以把你毙了。

  注12:我们小学的时候说出了一块砖头的35种用法,而且是因为课时不够的才停止的,不知道现在的小学生怎么样。

  注13:学任何技术,不只是,7分为自己,3分为国家,别去想着钱的问题,你真的有本事,你怕没钱赚?我踏妈最讨厌别人跟我说现在学计算机的人太多了,没钱途的。

  注14:不过如果想让别人也认同你是黑客,你必须有出众的技术,并且得到老一辈的认可。

  注15:有能力看原版的千万别买翻译的看,想象一下,最经典的C++ Primer竟然可以交给一个把Shell Programming翻译成外壳编程的人翻。







\end{CJK}
\end{document}


