\documentclass[12pt,a4paper]{article}
\usepackage{geometry}
\geometry{left=2.5cm,right=2.5cm,top=2.5cm,bottom=2.5cm}

\usepackage[utf8]{inputenc}
\usepackage{listings}
\usepackage{CJK}
\usepackage{xcolor}
\usepackage{graphicx}
\usepackage{indentfirst}


\begin{document}
\begin{CJK}{UTF8}{gbsn}
\title{判断唯一可译码}
\author{ismdeep}
\date{2013-10-17 10:56:58}


\maketitle

%设置listings
\lstset{numbers=left,
numberstyle=\tiny,
keywordstyle=\color{blue!100}, commentstyle=\color{red!50!green!50!blue!50},
frame=single,tabsize=4,showtabs=false,extendedchars=false,
rulesepcolor=\color{red!20!green!20!blue!20}
}

\newpage
\section{唯一可译码}
任意有限长的码元序列,只能被唯一地分割成一个个的码字,便称为唯一可译码。

\section{判断唯一可译码}
    将码C中所有可能的尾随后缀组成一个集合F,当且仅当集合F中没有包含任一码字,则可判断此码C为唯一可译变长码。如何构成集合F,可以如下进行。

首先,观察码C中最短的码字是否是其他码字的前缀。若是,将其所有的可能的尾随后缀排列出。而这些尾随后缀又可能是某些码字的前缀,再将由这些尾随后缀产生的新的尾随后缀列出。

然后再观察这些新的尾随后缀是否是某些码字的前缀,再将产生的后缀列出。

依次下去,直至没有一个尾随后缀是码字的前缀或没有新的尾随后缀产生为止。这样,首先获得由最短的码字能引起的所有尾随后缀。接着,按照上述步骤将次短的码字、….等等,所有码字可能产生的尾随后缀全部列出。

由此,得到由码C的所有可能的尾随后缀组成的集合F。

\end{CJK}
\end{document}



