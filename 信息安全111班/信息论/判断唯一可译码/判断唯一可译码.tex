\documentclass[12pt,a4paper]{article}
\usepackage{geometry}
\geometry{left=2.5cm,right=2.5cm,top=2.5cm,bottom=2.5cm}

\usepackage[utf8]{inputenc}
\usepackage{listings}
\usepackage{CJK}
\usepackage{xcolor}
\usepackage{graphicx}
\usepackage{indentfirst}
\usepackage{url}


\begin{document}
\begin{CJK}{UTF8}{gbsn}
\title{判断唯一可译码}
\author{ismdeep}
\date{2013-10-17 10:56:58}


\maketitle

%设置listings
\lstset{numbers=left,
numberstyle=\tiny,
keywordstyle=\color{blue!100}, commentstyle=\color{red!50!green!50!blue!50},
frame=single,tabsize=4,showtabs=false,extendedchars=false,
rulesepcolor=\color{red!20!green!20!blue!20}
}

\newpage
\section{唯一可译码}
任意有限长的码元序列,只能被唯一地分割成一个个的码字,便称为唯一可译码。

\section{判断唯一可译码}
    将码C中所有可能的尾随后缀组成一个集合F,当且仅当集合F中没有包含任一码字,则可判断此码C为唯一可译变长码。如何构成集合F,可以如下进行。

首先,观察码C中最短的码字是否是其他码字的前缀。若是,将其所有的可能的尾随后缀排列出。而这些尾随后缀又可能是某些码字的前缀,再将由这些尾随后缀产生的新的尾随后缀列出。

然后再观察这些新的尾随后缀是否是某些码字的前缀,再将产生的后缀列出。

依次下去,直至没有一个尾随后缀是码字的前缀或没有新的尾随后缀产生为止。这样,首先获得由最短的码字能引起的所有尾随后缀。接着,按照上述步骤将次短的码字、….等等,所有码字可能产生的尾随后缀全部列出。

由此,得到由码C的所有可能的尾随后缀组成的集合F。

\section{实现代码}
详细代码在src文件夹中。
或者访问网址:

短网址:\url{http://t.cn/zRVLbjO}

原网址:\url{https://github.com/ismdeep/public_documents/blob/master/%E4%BF%A1%E6%81%AF%E5%AE%89%E5%85%A8111%E7%8F%AD/%E4%BF%A1%E6%81%AF%E8%AE%BA/%E5%88%A4%E6%96%AD%E5%94%AF%E4%B8%80%E5%8F%AF%E8%AF%91%E7%A0%81/src/main.cpp}



\begin{lstlisting}[language=C++]
#include <iostream>
#include <stdlib.h>
#include <string.h>

using namespace std;

struct strings
{
    char *string;
    struct strings *next;
};


struct strings  Fstr, *Fh, *FP;
void outputstr(strings *str)
{                
         do
         {
                   cout<<str->string<<endl;
                   str = str->next;
         }while(str);
         cout<<endl;
}
inline int MIN(int a, int b)
{    return a>b?b:a;     }
inline int MAX(int a, int b)
{    return a>b?a:b;     }
#define length_a (strlen(CP))
#define length_b (strlen(tempPtr))
//判断一个码是否在一个码集合中,在则返回0,不在返回1
int comparing(strings *st_string,char *code)
{
         while(st_string->next)
         {
                   st_string=st_string->next;
                   if(!strcmp(st_string->string,code))
                            return 0;
         }
         return 1;
}
 
//判断两个码字是否一个是另一个的前缀,如果是则生成后缀码
void houzhui(char *CP,char *tempPtr)
{
         if (!strcmp(CP,tempPtr))
         {
                   cout<<"集合C和集合F中有相同码字:"<<endl
                            <<CP<<endl
                            <<"不是唯一可译码码组!"<<endl;
                   exit(1);
         }
         if (!strncmp(CP, tempPtr, MIN(length_a,length_b)))
         {                                            
                   struct strings *cp_temp;
                   cp_temp=new (struct strings);
                   cp_temp->next=NULL;
                   cp_temp->string=new char[abs(length_a-length_b)+1];                     
                   char *longstr;
                   longstr=(length_a>length_b ? CP : tempPtr);//将长度长的码赋给longstr
 
                   //取出后缀
                   for (int k=MIN(length_a,length_b); k<MAX(length_a,length_b); k++)
                            cp_temp->string[k - MIN(length_a,length_b)]=longstr[k];
                   cp_temp->string[abs(length_a-length_b)]=NULL;
                   //判断新生成的后缀码是否已在集合F里,不在则加入F集合
                   if(comparing(Fh,cp_temp->string))
                   {
                            FP->next=cp_temp;
                            FP=FP->next;
                   }
         }
}


int main()
{
         cout<<"\t\t唯一可译码的判断!\n"<<endl;
         struct strings  Cstr, *Ch, *CP,*tempPtr;   
    Ch=&Cstr;
         CP=Ch;
    Fh=&Fstr;
         FP=Fh;
         char c[]="C :";
    Ch->string=new char[strlen(c)];
    strcpy(Ch->string, c);      
    Ch->next=NULL;
    char f[]="F :";
    Fh->string=new char[strlen(f)];
    strcpy(Fh->string, f);
    Fh->next=NULL;
//输入待检测码的个数
         int Cnum;
         cout<<"输入待检测码的个数:";
         cin>>Cnum;
         cout<<"输入待检测码"<<endl;
         for(int i=0; i<Cnum; i++)
    {
                   cout<<i+1<<" :";
                   char tempstr[10];
                   cin>>tempstr;  
                   CP->next=new (struct strings);
                   CP=CP->next;
                   CP->string=new char[strlen(tempstr)] ;
                   strcpy(CP->string, tempstr);
                   CP->next = NULL;
    }
         outputstr(Ch);
    CP=Ch;
    while(CP->next->next)
    {
                   CP=CP->next; 
                   tempPtr=CP;            
                   do
                   {
                            tempPtr=tempPtr->next;
                            houzhui(CP->string,tempPtr->string);
                   }while(tempPtr->next);
    }
    outputstr(Fh);
         struct strings *Fbegin,*Fend;
    Fend=Fh;
    while(1)
    {
                   if(Fend == FP)
                   {
               cout<<"是唯一可译码码组!"<<endl;
                            exit(1);
                   }
                   Fbegin=Fend;
                   Fend=FP;
                   CP=Ch;             
                   while(CP->next)
                   {
            CP=CP->next;
                            tempPtr=Fbegin;
                            for(;;)
                            {
                                     tempPtr=tempPtr->next;
                                     houzhui(CP->string,tempPtr->string);
                                     if(tempPtr == Fend)
                                               break;
                            }
                   }       
                   outputstr(Fh);//输出F集合中全部元素
    }
	return 0;
}



\end{lstlisting}


\end{CJK}
\end{document}



