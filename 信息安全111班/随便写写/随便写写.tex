\documentclass[12pt,a4paper]{article}
\usepackage{geometry}
\geometry{left=2.5cm,right=2.5cm,top=2.5cm,bottom=2.5cm}

\usepackage[utf8]{inputenc}
\usepackage{listings}
\usepackage{CJK}
\usepackage{xcolor}
\usepackage{graphicx}


\begin{document}
\begin{CJK}{UTF8}{gbsn}
\title{随便写写--2013-10-10信息安全131班交流用}
\author{ismdeep}
\date{2013-10-09 23:37:21}


\maketitle

%设置listings
\lstset{numbers=left,
numberstyle=\tiny,
keywordstyle=\color{blue!100}, commentstyle=\color{red!50!green!50!blue!50},
frame=single,tabsize=4,showtabs=false,extendedchars=false,
rulesepcolor=\color{red!20!green!20!blue!20}
}


\section{语言基础}
核心概念:语言只是表象,算法才是核心。数据结构决定了使用什么样的算法。

其实语言都差不多,语言的分类,貌似就差不多那么几个吧。lisp类型的,asm那种的mov的,c这种面向过程的,C++,java这种面向对象的。


\section{学习路线}
一万小时定律

人们眼中的天才之所以卓越非凡,并非天资超人一等,而是付出了持续不断的努力。只要经过1万小时的锤炼,任何人都能从平凡变成超凡。

个人不怎么喜欢做一些GUI之类的界面,因为没有时间去学习这样一些没有什么意思的东西。

如果真正学习算法的话,需要非常长的时间。

\section{日常}
OSC,逛社区,看博客,写博客。

在这样的环境下熏陶吧。

\section{关于信息安全}
个人对于信息安全专业以及信息安全行业的了解并不是非常多。大部分时间都花在了算法和数据结构的学习上面。

信息安全的话,貌似网上可以看到很多的信息安全大会。但是我之前看到过一些信息安全大会,其实并不像你想象中的那样。

有一些是某个安全公司,比如360啊。实际是在推广自己的安全产品而已。那就没有什么意思了。。。

如果英语水平不错的话,可以去看一些国外的黑客大会什么的。还是很不错的。。比如美国就有,话说每年都有的,黑帽黑客大会。

\section{大学生活}
两年时光过去了,自己的生活上有一些改变。

大一时光$>>$

上学期:大部分时光都在编程和乱逛中过去了。

下学期:周末时光花在了集训和准备跟随高年级去金华参加全国邀请赛。主要是去打酱油了。

大一暑假$>>$

集训集训还是集训。做了很多题目。整天能看到的人,就只有那么几个人。

大二时光$>>$

上学期:大部分时光都在玩游戏。本来是有两个实验室可以去的。

下学期:去了长沙参加全国邀请赛。0题惨败,对面清华大学,冠军队。参加全球总决赛。看着羡慕,但是自己实力这样,只能看着。

大三时光$>>$

刚开始。两场全国决赛。

\section{关于挂科}
每个学期,总会有人会挂科,甚至大一都去不了大二都有。

我也是经常挂科,现在挂了2科,补考没过。

现在信息论与编码,院长的课,点名要挂我科。纠结了,最近真的很忙啊。


\end{CJK}
\end{document}



