\documentclass[12pt,a4paper,landscape]{article}
\usepackage{geometry}
\geometry{left=2.5cm,right=2.5cm,top=2.5cm,bottom=2.5cm}

\usepackage[utf8]{inputenc}
\usepackage{listings}
\usepackage{CJK}
\usepackage{xcolor}
\usepackage{graphicx}


\begin{document}
\begin{CJK}{UTF8}{gbsn}
\title{使用sshfs挂载远程Linux文件系统}
\author{ismdeep}
\date{Date}


\maketitle

%??listings
\lstset{numbers=left,
numberstyle=\tiny,
keywordstyle=\color{blue!100}, commentstyle=\color{red!50!green!50!blue!50},
frame=single,tabsize=4,showtabs=false,extendedchars=false,
rulesepcolor=\color{red!20!green!20!blue!20}
}


在使用Linux的时候,很多时候我们需要挂载远程的目录来工作,比如笔者自己就喜欢挂载远程sourceforge上的数据目录来获取一些源代码,或者编辑个人主页;还有时候需要挂载学校内某个服务器的目录来处理管理网站的处理。这些远程目录的挂载都可以通过sshfs这个软件来实现。

下面就说一下常用ubuntu系统怎么使用sshfs吧。

ubuntu系统下安装sshfs只需要使用下面命令便可以完成。

\begin{lstlisting}[language=bash]
sudo apt-get install sshfs
\end{lstlisting}

sshfs选项很多啊。
\begin{lstlisting}[language=bash]
general options:
    -o opt,[opt...]        mount options
    -h   --help            print help
    -V   --version         print version

SSHFS options:
    -p PORT                equivalent to '-o port=PORT'
    -C                     equivalent to '-o compression=yes' #启用压缩,建议配上
    -F ssh_configfile      specifies alternative ssh configuration file #使用非默认的ssh配置文件
    -1                     equivalent to '-o ssh_protocol=1' #不要用啊
    -o reconnect           reconnect to server               #自动重连
    -o delay_connect       delay connection to server
    -o sshfs_sync          synchronous writes
    -o no_readahead        synchronous reads (no speculative readahead) #提前预读
    -o sshfs_debug         print some debugging information
    -o cache=BOOL          enable caching {yes,no} (default: yes) #能缓存目录结构之类的信息
    -o cache_timeout=N     sets timeout for caches in seconds (default: 20)
    -o cache_X_timeout=N   sets timeout for {stat,dir,link} cache
    -o workaround=LIST     colon separated list of workarounds
             none             no workarounds enabled
             all              all workarounds enabled
             [no]rename       fix renaming to existing file (default: off)
             [no]nodelaysrv   set nodelay tcp flag in sshd (default: off)
             [no]truncate     fix truncate for old servers (default: off)
             [no]buflimit     fix buffer fillup bug in server (default: on)
    -o idmap=TYPE          user/group ID mapping, possible types are:  #文件权限uid/gid映射关系
             none             no translation of the ID space (default)
             user             only translate UID of connecting user
    -o ssh_command=CMD     execute CMD instead of 'ssh'
    -o ssh_protocol=N      ssh protocol to use (default: 2) #肯定要2的
    -o sftp_server=SERV    path to sftp server or subsystem (default: sftp)
    -o directport=PORT     directly connect to PORT bypassing ssh
    -o transform_symlinks  transform absolute symlinks to relative
    -o follow_symlinks     follow symlinks on the server
    -o no_check_root       don't check for existence of 'dir' on server
    -o password_stdin      read password from stdin (only for pam_mount)
    -o SSHOPT=VAL          ssh options (see man ssh_config)

Module options:

[subdir]
    -o subdir=DIR       prepend this directory to all paths (mandatory)
    -o [no]rellinks     transform absolute symlinks to relative

[iconv]
    #字符集转换,对我这种UTF8控,默认已经是最好的
    -o from_code=CHARSET   original encoding of file names (default: UTF-8)
    -o to_code=CHARSET      new encoding of the file names (default: UTF-8)
\end{lstlisting}


实际使用

远程挂载

\begin{lstlisting}
sshfs root@192.168.1.101:/home/files /home/files_remote
\end{lstlisting}

卸载
\begin{lstlisting}
fusermount -u /home/files_remote
\end{lstlisting}




\end{CJK}
\end{document}


